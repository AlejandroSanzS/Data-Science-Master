%%%%%%%%%%%%%%%%%%%%%%%%%%%%%%%%%%%%%%%%%%%%%%%%%%
%%%%% - Paquetes empleados en mi Documento - %%%%%
%%%%%%%%%%%%%%%%%%%%%%%%%%%%%%%%%%%%%%%%%%%%%%%%%%

\usepackage[utf8]{inputenc}
\usepackage[spanish]{babel}
\usepackage[left=2.5cm,right=2.5cm,top=2.5cm,bottom=2.5cm]{geometry}
\usepackage{fancyhdr}
\usepackage[footnotesize]{caption}
\usepackage{subcaption}
\usepackage{blindtext}
\usepackage{enumitem}
\usepackage{graphicx}
\usepackage{adjustbox}
\usepackage{tabularx}
\usepackage[maxbibnames=99,backend=bibtex,sorting=none]{biblatex}
\usepackage{tcolorbox} % para que te haga tablas del indice molonas
\usepackage{hyperref}
\hypersetup{
	pdfnewwindow=true,              % Links en una nueva ventana
	linktoc=page,                   % Dónde aparece el link (none,section,page,all)
	colorlinks=true,                % false: Links en cajas; true: Links en colores
	linkcolor=cyan,                 % Color de los links internos
	citecolor=blue,                % Color de los links de la bibliografía
	filecolor=magenta,              % Color de los links de archivos
	urlcolor=blue                    % Color de los links externos
}

%\usepackage[urw-garamond]{mathdesign} % http://www.ctan.org/tex-archive/fonts/mathdesign/

\usepackage{amsmath}
\usepackage{amsfonts}
\usepackage{amssymb}
\usepackage{mathtools}
\usepackage{nccmath}

\newcommand\e{\text{e}}
\renewcommand\d{\text{d}}

\graphicspath{{./Imagenes/}}
\DeclareGraphicsExtensions{.pdf,.jpeg,.png,.jpg}

%hace referencia a la carpeta donde tiene que cargar las imagenes,por eso en el documento raiz llama a la imagen mediante el nombre y la extension directamente

